%===================================== CHAP 1 =================================

\chapter{Introduction}

\section{Background}
Recent years has seen an intense research into plasmonics and applications, therein the development of the field of metamaterials and metasurfaces[REF]. The main applications of nanoplasmonics result from the ability to confine the electromagnetic field into subwavelength volumes and field enhancement. This manipulation of light have applications such as photothermal cancer treatment, in vivo bio-imaging, catalysis, thermal emitters, nonlinear optics, photodetectors, and in engineering of radiative decay\cite{Au_nanorods_review}\cite{Schuller_plasmonicapplications_review}. In photovoltaics, plasmonic nanoparticles have shown potential in boosting performance of solar cells while reducing the material costs \cite{Trugler_metallicnanoparticles}\cite{Green_plasmonic_solarcells}.

%Plasmonic nanostructures is an active field of research due to their fascinating optical, electrical and chemical properties and due to their current and potential practical applications\cite{Trugler_metallicnanoparticles}\cite{Stockman:11}. Plasmonics enables the ability to concentrate light and/or produce high local-field enhancements in subwavelength structures. This manipulation of light have applications such as photothermal cancer treatment, in vivo bio-imaging, catalysis, thermal emitters, metamaterials, nonlinear optics, photodetectors, and in engineering of radiative decay\cite{Au_nanorods_review}\cite{Schuller_plasmonicapplications_review}. In photovoltaics, plasmonic nanoparticles have shown potential in boosting performance of solar cells while reducing the material costs \cite{Trugler_metallicnanoparticles}\cite{Green_plasmonic_solarcells}.

Researchers are investigating the optical properties of an increasing variety of metallic nanostructures with the hope of effectively concentrating light into nanoscale volumes. In resonant structures, timevarying electric fields associated with light waves exert a force on the gas of negatively charged electrons inside a metal and drive them into a collective oscillation. At specific frequencies this oscillation is resonantly driven to produce a very strong charge displacement and associated (light) field concentration, known as a localized surface plasmon resonance (LSPR). LSPRs and surface plasmon polaritons (SPP) confine the field near a particle or interface, which has applications ranging from sub-diffraction-limit imaging, through nanophotonics communications to photovoltaics. The design go well beyond simple non-interacting nanoparticles to e.g. structures such as plasmonic nano-antennas[REF].

Plasmonics has also been a fundamental sub-unit in the design of the "meta-atom" in metamaterials, not only supplying the negative dielectric function through metallic rods, but also supplying an effective negative permeability through e.g. split ring resonator designs[REF]. Metamaterials unfortunately suffer from large losses and are extremely complicated to manufacture, but recently metasurfaces exploring a two-dimenional lattice of resonator units has become a promising field of research [Capasso].
On the other hand, any nanostructure made out of e.g. semiconductors can be used as quantum dots (e.g. GaSb, PbSe dots), highly absorbing solar cell units (e.g. Si nanopillars, GaAs nanopillars [se Fimland og Wemanns gruppe]). Modelling of such nanostructures has traditionally been limited to exploring the generalized effective medium theory. However, since effective medium theory relies on structures being much smaller than the wavelength of light (effectively in the quastistatic approximation), this approach is expected to break down for larger structures[REF].
%Propagating surface plasmons, known as surface plasmon polaritons (SPP), and localized surface plasmons (LSP) confine the field near an interface or near a particle, and this confinement has applications ranging from sub-diffraction-limit imaging, through nanophotonics communications to photovoltaics. The design go well beyond simple non-interacting nanoparticles to e.g. structures such as well-designed plasmonic nano-antennas.



%LSPRs in nanostructures of noble metals can be excited by electromagnetic radiation in the optical regime, and optical characterization methods can therefore be applied. 
Since the plasmon excitation is polarization dependent\cite{maier}, ellipsometry is a natural candidate for sample characterization as it exploits the polarization aspect of light. Spectroscopic ellipsometry is an established non-invasive method of retrieving valuable amplitude and phase information from metal-dielectric interfaces of thin films, from which one can determine the effective dielectric function and how it relates to the material nanostructure and define exactly the system's plasmonic characteristics\cite{hans_arwin_reviewarticle}. 

Traditionally, the optical response has been modelled by a stack of plane layers. Nanostructured materials (with or without plasmonic nanoparticles) could be modelled in terms of the generalized effective medium theory, whereas the optical response of the layer can be modelled by an anisotropic layer[REF]. The complexity of the structures, and that the dimensions are not strictly subwavelength makes the latter approach insufficient. This is where computational electromagnetics can help out. Furthermore, Mueller matrix ellipsometry covers the complete polarization response from such complex nanostructures. The modelling of the full Mueller matrix from nanostructured samples is the topic of this thesis.


There are several different methods in computational electromagnetics, each with their own advantages and disadvantages depending on the problem at hand. The discrete dipole approximation (DDA) computes electromagnetic scattering and absorption by targets of arbitrary shape, but is limited by requiring a interdipole separation small compared the structural lengths in the target and to the wavelength\cite{DDA}. In the finite difference time domain (FDTD) method, Maxwell's equations in their differential form are discretized in space and time and the time evolution of electromagnetic near-fields is calculated directly\cite{FDTD}. FDTD is an attractive method due to being relatively easy to implement for specific problems, however, resolving curved and triangular geometries is challenging. The finite element method (FEM) is another full-wave differential equation solver, it discretizes the system volume into smaller, simpler parts that are numerically easier to solve. These subsets are then re-assembled into a larger system of equations that models the entire problem\cite{FEM_in_EM_jianming_jin}. FEM is a mature method that is efficient and unconditionally stable\footnote{That is, if a wavelength step-size no matter how large can be used without resulting in the type of nonsense results associated with a numerical instability.}\cite{CEM_course}, and solving Maxwell's equations in the frequency domain allows direct comparison with ellipsometric characterization of experimental data. Its most attractive feature, however, is its ability to handle complicated geometries with relative ease, making FEM the numerical method of choice for this thesis. 

% Choosing the right technique is important, as choosing the wrong one may result in incorrect results or have excessively long computation time\text{\color{red}fjern setning?}.
%Something FEM. Full wave solution. Retardation effects. Can model intricate geometries such as curved objects more accurately than.
%localized surface plasmon resonances (LSPR) of noble metal nanoparticles, originating from the collective oscillations of conduction electrons, are strongly dependent on the size, shape, composition and the surrounding environment. %http://pubs.acs.org/doi/pdf/10.1021/jp500638u

%The plasmon resonance of an ensemble of coupled particles depends on interparticle distance, and its wavelength position is influenced by particle charge and the refractive index of the particle's immediate environment.
%https://link.springer.com/content/pdf/10.1007%2Fs11468-010-9130-2.pdf !!

\section{Problem formulation}
This thesis will investigate methods to simulate the optical properties of periodic metallic nanostructures at oblique angle of incidence using COMSOL Multiphysics, a commercial finite element method software. There will be a focus on replicating structures that have been fabricated in NTNU NanoLab and characterized with spectroscopic ellipsometry. If the model is verifiable through comparison with experimental data, it will be applied to simulate other structures. Simulation results for these nanostructures will be given as a function of wavelength, azimuthal angle and polarization of the incident light, including in the form of spectroscopic Mueller matrices.
% for a spectral range in full azimuthal rotation for both longitudinal and transverse polarizations,

%There will be a focus on structures that are possible to fabricate in NTNU NanoLab and characterize with optical methods in a laboratory. Simulation results for these nanostructures will be reported in full azimuthal rotation for both longitudinal and transverse polarizations, including in the form of spectroscopic Mueller matrices.

\text{\color{red}OK??}
%in particular gold hemispheroidal particles on a glass substrate\cite{Brakstad:15}\cite{Kildemo2017} and densely packed inclined GaSb cones\cite{gasbcones}. An attempt should be made to create a user-friendly model 

%Preferably, a successful model should be applicable to other nanostructures 

%Will report simulation results for the plasmonic properties of these nanostructures under both longitudinal and transverse polarizations



%I denne oppgaven skal en studere forskjellige metodikker for å kunne modellere optiske egenskaper til metamaterialer og nanoplasmonikk materialer

%Why optimization
\section{Thesis overview}
In the following chapter, an introduction to electromagnetic theory will be given; the polarization of light, Stokes-Mueller formalism, ellipsometry, and electromagnetic interaction with matter including surface plasmons and diffraction grating, before finally introducing the finite element method. Chapter 3 will present the various fabricated nanostructures to be modeled as well as a brief summary of their experimental background. The commercial FEM software COMSOL Multiphysics will be presented in chapter 4, beginning with a general overview of the program before delving further into the wave optics module used to simulate electromagnetic wave propagation. The results and discussion in chapter 5 will be divided into three main parts; a detailed explanation of the modeling process in COMSOL will be given first; next, measures taken to optimize the model and their effects on performance and accuracy are presented; lastly, the simulation results and analysis of each of the four samples are given. Conclusions and recommendations for future work will be given in chapter 6. 





%\cleardoublepage