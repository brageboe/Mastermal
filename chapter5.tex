%===================================== CHAP 5 =================================

\chapter{Conclusion and outlook}

\subsection*{Conclusion}
The primary goal of the work on this thesis - developing a functioning COMSOL model of infinite periodic plasmonic surfaces and testing its validity against established experimental work - has been fulfilled. The simulation results were found in good correspondence with experimental data from three variations of hemispheroidal gold particles on glass substrates. Time would allow to test the model on only one additional system, unfortunately, a nanostructure of densely packed tilted GaSb cones. However, there is confidence in the finished state of the COMSOL model, which can be used as a template for modeling the optical response of other structures with 2D periodicity of both rectangular and hexagonal lattices. Minor adjustments such as mesh setup will likely be necessary when implementing a new system, depending on structure geometry. (The model allow for simulation in a user-specified spectral region and supports full azimuthal rotation.\text{\color{red}fjern siste setning?})

%COMSOL models have been developed for [Au samples],. This model was further used to succesfully model a very different sample of tilted GaSb cones in a hexagonal lattice. These files can effectively be used as template files for simulating other similar structures. It should in principle be feasible for a user with limited knowledge in COMSOL to use these templates to simulate other ordered nanostructure that can be modeled as a unit cell in an infinite periodic 2D (rectangular or hexagonal) lattice. Minor adjustments such as mesh setup can be necessary, depending on structure geometry. The models allow for simulation in a user-specified spectral region and supports full azimuthal rotation.

COMSOL is effective in solving the full-wave solution for single wavelengths, however, solving for a large sequence of several hundred wavelength iteration was found to be very time consuming. Efforts put into optimizing the model drastically reduced both the memory requirement and computation time; for each increasing wavelength iteration they were found to decrease exponentially. Computational cost is heavily dependent on the size of the unit cell compared to the wavelength, which ultimately decides the amount of finite elements the computer has to solve. %Implementing perfectly matched layers on the domain boundaries ....

Results from these simulations can be represented as ellipsometry angles (both amplitude and phase differences), Mueller matrices, reflection coefficients, reflectance, heat dissipation and field distributions. There is confidence in the possibility to extract phase shift information from S-parameters, although this research was not fully completed.

Hvilken ny info har vi funnet vha COMSOL modellene?

\subsection*{Outlook}
%\text{\color{red}improve}Optimization is a never ending story. The simulations done here have not fully exploited symmetry of the nanostructures. By building the geometry only to its symmetry point, one may greatly reduce the computational cost. E.g. splitting sample 5B unit cell in half would likely let us simulate lower wavelengths than the 250 nm done here, as it would effectively half the amount of elements to solve. I don't, however, think this is trivial to implement in COMSOL. In short, outlook: exploit symmetry to create computationally less expensive models and compare results with full 3D models.

We have shown that the COMSOL models work for a full 3D representation of each sample unit cell. However, there are symmetries in the nanostructures that have not been exploited. For example, by building the geometry only to its symmetry point and simulating the omitted geometry using symmetry boundary conditions, one may potentially reduce the computational cost by a significant amount. The solution may even be more accurate as a much denser mesh can be used. For sample 5B, this could mean a successful solution of wavelengths much lower than $250$ nm, while the tilted GaSb cones would come closer to its $24$ eV goal. Furthermore, COMSOL's support for cluster computing remains unexplored, which could result in an extreme reduction in computation time. %Optimization is never-ending.


\cleardoublepage