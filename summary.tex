\clearpage
\pagenumbering{roman} 				
\setcounter{page}{1}

\pagestyle{fancy}
\fancyhf{}
\renewcommand{\chaptermark}[1]{\markboth{\chaptername\ \thechapter.\ #1}{}}
\renewcommand{\sectionmark}[1]{\markright{\thesection\ #1}}
\renewcommand{\headrulewidth}{0.1ex}
\renewcommand{\footrulewidth}{0.1ex}
\fancyfoot[LE,RO]{\thepage}
\fancypagestyle{plain}{\fancyhf{}\fancyfoot[LE,RO]{\thepage}\renewcommand{\headrulewidth}{0ex}}

\section*{\Huge Abstract}
\addcontentsline{toc}{chapter}{Summary}	

\noindent 
This thesis is focused on modelling the full optical response of periodic nanostructures with a two-dimensional lattice using the commercial software COMSOL Multiphysics, which is based on the finite element method. Initially being very computationally demanding, the model has undergone several optimization measures which greatly reduced computation time and memory requirements. The validity\text{\color{red}accuracy?} of the model was successfully tested against established experimental work on plasmonic nanostructures; three different variants of gold hemispheroidal particles in square and rectangular lattices on a SiO$_2$ substrate. The samples exhibit a surprisingly complex spectroscopic Mueller matrix response given their reasonably simple lattices, including localized surface plasmon resonances (LSPR) and a strong dependency on Rayleigh anomalies. The verified model was further used to simulate a sample of densely packed tilted GaSb cones for photon energies beyond its corresponding experimental work, revealing a strong polarization coupling around $7.5$ eV. Simulation run-time was found to be excessive when iterating over a large range of wavelengths. Further optimization measures have been suggested that could potentially improve performance which in turn would open up possibilities of modeling more complex and computationally demanding systems.


%The validity\text{\color{red}accuracy?} of the model was successfully tested against established experimental work on samples with surprisingly complex measured optical response given their reasonably simple lattices of plasmonic nanostructures; three different variants of gold hemispheroidal particles in square and rectangular lattices on a SiO$_2$ substrate. 

%\begin{itemize}
%    \item Introduction. Hvorfor er ordered plasmonic nanostructures interessant. Hvorfor simulere dem.
%    \item Simuleringsprosess. FEM. hvorfor FEM. 
%\end{itemize}



\clearpage